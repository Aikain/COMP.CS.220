Kurssin suoritus on mahdollisesti jokseenkin vajaavainen ja ei lähelläkään sitä
miihen olisi voinut pystyä mikäli ajanhallinta olisi vain ollut kunnossa.
Kurssin toteuttaminen muutaman päivän ryppäinä useiden viikkojen tai jopa
kuukausien tauolla aiheutti jonkin verran ylimääräistä työtä, mutta asioiden
ollessa muuten melko yksinkertaisia, nopealla vilkaisulla pääsi jälleen
kärrylle.

Opiskelussa käytetyt kurssit toivat hyvin asioita esille ja olivat mukavan
yksinkertaisia ja kevyitä tehdä. Lähes 10 vuoden ohjelmointiosaamisen ansioista
asioiden oppiminen onnistui lisäksi pitkälti ongelmitta. Kehittämisessä on
lisäksi selvästi nähtävissä ero siihen, mitä kehittäminen oli vuonna 2013 kun
tehtiin Microsoft Phonelle.

Laaja harjoitustyö on selvästi keskeneräinen ja kehitys tulee jatkumaan aina
kun löytyy vapaata aikaa ja innostusta. Projektiin on jo mahdollisesti löytynyt
graaffikkokin niin saa huomattavasti näyttävämpää jälkelä kuin itse nopeasti
AI:n avustuksella. Tavoitteena on päästä testaamaan sovellusta seuraavalla
leirillä jolla osallistun kun sovellus on hiukan liian vajaavainen, että olisi
voinut testata leirillä, jolla olen parhaillaan tätä yhteenvetoa kirjoittaessa.

Oheiseen tauluun on koottu yhteenveto aiempien lukujen tuntikirjanpidoista.

\begin{table}[H]
    \centering
    \label{tab:working-hours-summary}
    \begin{tabular*}{\linewidth}{@{\extracolsep{\fill}} l r }
        \textbf{Tekeminen} & \textbf{Määrä} \\
        \hline
        Aloitus/oppimispäiväkirja & 15h 05m \\
        Opiskelu: Android Basics in Kotlin -kurssi & 36h 25m \\
        Opiskelu: Android Basics with Compose -kurssi & 16h 45m \\
        Opiskelu: muu & 3h 45m \\
        Harjoitustehtävät & 23h 30m \\
        Harjoitustyö & 17h 45m \\
        \hline
        \multicolumn{1}{l}{\textbf{Yhteensä}} & \textbf{113h 15m} \\
    \end{tabular*}
\end{table}
