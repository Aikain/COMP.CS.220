Useamman vuoden ajan olen osallistunut Mobiiliohjelmointi kurssille tavoitteena
saada lopulta toteutettua Mafioso-seurapeli puhelimelle. Kuitenkin kurssin
ollessa vapaasti valittavissa opinnoissa, on se jäänyt aiemmin aina
alhaisemmalle prioriteetille. Nyt kun muut opinnot diplomityötä lukuunottamatta
on tehty, on viimein hyvä hetki suorittaa kurssi.

Ensimmäisessä luvussa käsitellään kurssin alkupuolen valmisteluja, jotka
sisältävät mm. versionhallinan käyttöönoton, oppimispäiväkirjan alustamisen ja
automatisoinnin sekä Android Studion asennuksen.

Toisessa luvussa on kerrottu opiskeluun liittyvistä asioista, joihin kuuluu
lievä alustus minkä takia valittu mitä opiskeltavaksi, verkkokurssien
läpikäynti ja muiden yksittäisen asioiden läpikäynti.

Kolmannessa luvussa on käyty läpi kaikki harjoitustyöt.

Neljäs luku esittelee harjoitustyön idean, kertoo kevyesti harjoitustyön
suunnittelusta, käy läpi tekemisen ja esittelee lopulta kurssin aikarajoen
puittessa valmistuneen sovelluksen.
