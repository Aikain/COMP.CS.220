% TODO START

Tähän tulee tietoa miten kurssin asioita opiskeltiin kurssin edetessä yms.

\section{Ohjelmointikielet}

\subsection{Java}

\subsection{Kotlin}

* https://kotlinlang.org/

Miksi?

* Kiinnostaa opetella

* Jetbrains

* Onko parempi kuin java vai vaan mielipide?

* multiplatform?

* google suosittelee?

Muuta:

* https://developer.android.com/kotlin/style-guide

% TODO END

\section{Kurssit}

Ensimmäisenä etsiessä mahdollisia MOOC:eja tai muita vastaavia ratkaisuja
opiskelun tueksi päätyy suoraan Android:n sivuille
\parencite{AndroidDevelopersTrainingCourses} selaamaan kursseja.
Mobiiliohjelmointi-kurssin alkupuolella 25.01.2023 sivulla oli tarjolla kaksi
kurssia ''New! Android Basics with Compose''
\parencite{AndroidDevelopersAndroidBasicsInKotlin} ja ''Android Basics in
Kotlin'' \parencite{AndroidDevelopersAndroidBasicsWithCompose}, kuten
WaybackMachinen
\parencite{AndroidDevelopersTrainingCoursesWaybackMachine202301} avulla
saadusta kuvasta
\ref{fig:android-developers-training-courses-2023-01-24-wayback-machine}
nähdään. Näistä ensimmäisessä kuitenkin lukee kurssin olevan vielä työnalla,
joten tämä tekee valinnasta sinääli yksinkertaisen kun tarkkaa aikataulua
ensimmäisen kurssin valmistumiselle ei ole.

\begin{figure}[h!]
  \includegraphics[width=\textwidth]{figures/android-developers-training-courses-2023-01-24-wayback-machine.png}
  \caption{Kuvankaappaus Android Developers sivun tarjolla olevista kursseista 24.01.2023}
  \label{fig:android-developers-training-courses-2023-01-24-wayback-machine}
\end{figure}

Osittaiseksi epäonneksi ja huonon muistin ansiosta ei tullut kertaakaan
tarkistettua toisen kurssin tilaa vaan tein tuon aloitetun kurssin loppuun
asti. Kun kurssi tuli tehtyä loppun asti, lähdin katselemaan onko jotain muita
kursseja joita voisi vielä olla hyvä tehdä. Tällöin huomaan muutoksen, joka on
tapahtunut noissa kursseissa. Wayback Machinen perusteella muutos on tapahtunut
helmi-maaliskuun vaihteessa
\parencite{AndroidDevelopersTrainingCoursesWaybackMachine202302}
\parencite{AndroidDevelopersTrainingCoursesWaybackMachine202303}. Muutoksen
myötä ''Android Basics with Compose''-kurssi ei ole enää keskeneräinen ja
kirjoitushetkellä kyseinen kurssi onkin jopa suositeltu kurssi, kuten
kuvankaappauksesta \ref{fig:android-developers-training-courses-2023-05-23}
nähdään.

\begin{figure}[h!]
  \includegraphics[width=\textwidth]{figures/android-developers-training-courses-2023-05-23.png}
  \caption{Kuvankaappaus Android Developers sivun tarjolla olevista kursseista 24.01.2023}
  \label{fig:android-developers-training-courses-2023-05-23}
\end{figure}

''Android Basics in Kotlin''-kurssin yhteydessä on useamman kerran törmännyt
vanhentuneisiin riippuvuuksiin tai deprekoituneisiin asioihin, joten uudempi
kurssi tuntuu sen puolesta houkuttevalta. Lisäksi kurssilla vaikuttaisi olevan
uuttakin asiaa verrattuna edelliseen kurssiin; nimittäin Compose. Käyn läpi
kurssin ensimmäisen osion pitkälti hypäten jo tietämieni asioiden yli, mutta
käymällä kuitenkin huolella läpi kaikki uudet asiat. Ensimmäisen osion pohjalta
päätän tehdä kyseisen kurssin myös kokonaan päällekkäisyyksistä huolimatta.

\subsection{Android Basics in Kotlin -kurssi}

Seuraavissa luvuissa on käyty läpi ''Android Basics in Kotlin''-kurssin
tekemistä.

Kurssi käy läpi Kotlinin ja Androidin perusteet. Kurssilla käytetään UI:n
tekemiseen Views:ejä. Lähtövaatimuksissa kurssille ei juurikaan ole vaan
ohjelmointikin opetetaan alusta alkaen.

\subsubsection{25.01.2023}

% TODO START

Unit-1 https://developer.android.com/courses/android-basics-kotlin/unit-1

\subsubsection{26.01.2023}

Unit-2 https://developer.android.com/courses/android-basics-kotlin/unit-2

* Joidenkin asioiden muistaminen hidastaa, esim. android:textAppearance=''?attr/textAppearanceHeadline6''
* Riippuvuudet liian vanhoja -> pitää päivittää -> aiheutuuu uusia ongelmia
* Moni asioista korjaantuu ajamalla uusiksi
* https://github.com/android/android-test/issues/1589\#issuecomment-1316971699

Unit-3 https://developer.android.com/courses/android-basics-kotlin/unit-3

* Android Studiolla saa kätevästi lisätty iconeita

Lopetettu ekan osan loppuun

\subsubsection{12.05.2023}

Jatkettu unit 3, toinen osa

* Android Studio tarjoaa helposti riippuvuuksien yms. päivittämistä
  * Tästä aiheutuu kuitenkin liian ongelma jos liian tarkasti seuraa ohjeita
%  * https://developer.android.com/jetpack/androidx/releases/lifecycle#2.6.0-alpha05
  > Transformations is now written in Kotlin. This is a source incompatible change for those classes written in Kotlin that were directly using syntax such as Transformations.map - Kotlin code must now use the Kotlin extension method syntax that was previously only available when using lifecycle-livedata-ktx. When using the Java programming language, the versions of these methods that take an androidx.arch.core.util.Function method are deprecated and replaced with the versions that take a Kotlin Function1. This change maintains binary compatibility. (I8e14f)

\subsubsection{13.05.2023}

* tools:text <-- kätevä devatessa, mutta tän olemassa olon unohtaa jos ei näe sitä esimerkeissä

\begin{lstlisting}[
  basicstyle=\small,
  label={lst:TODO},
  language=Kotlin,
]
private val _price = MutableLiveData<Double>()
val price: LiveData<String> = Transformations.map(_price) {
  NumberFormat.getCurrencyInstance().format(it)
}
\end{lstlisting}

vs

\begin{lstlisting}[
  basicstyle=\small,
  label={lst:TODO},
  language=Kotlin,
]
private val _price = MutableLiveData<Double>()
val price: LiveData<String> = _price.map {
  NumberFormat.getCurrencyInstance().format(it)
}
\end{lstlisting}

* testImplementation vs androidTestImplementation
* https://stackoverflow.com/a/48951686/8380119

* https://github.com/google-developer-training/android-basics-kotlin-cupcake-app/issues/87

* https://github.com/google-developer-training/basic-android-kotlin-training-sports/pull/6

* Resizable emulator: https://developer.android.com/studio/run/resizable-emulator
* Vaihto unfolded foltable -> tablet erittäin hidas

Viimeisessä projektissa ''ongelmat'':
* Unohtu miten tehdä: android:text=''@\{@string/subtotal(viewModel.subtotal)\}''
  * https://stackoverflow.com/a/39354769/8380119
* Unohtu miten tehdä: findNavController\(\).navigate\(R.id.action_accompanimentMenuFragment_to_checkoutFragment\)

* https://github.com/google-developer-training/android-basics-kotlin-lunch-tray-app/pull/6
* https://github.com/google-developer-training/android-basics-kotlin-lunch-tray-app/pull/31
* https://github.com/google-developer-training/android-basics-kotlin-lunch-tray-app/pull/43

* https://stackoverflow.com/a/74955089/8380119
* androidx.fragment:fragment-testing 1.5.0 -> 1.6.0-rc01

* Virheet koodissa:
  * Yritin olla käyttämättä ''!!''
    * mutable valuen increase yms. pakottaa käyttää ''!!''?

\begin{lstlisting}[
  basicstyle=\small,
  label={lst:TODO},
  language=Kotlin,
]
if (_subtotal.value != null)
  _subtotal.value = _subtotal.value!! - previousAccompanimentPrice
\end{lstlisting}

  * alustin \_subtotal ja \_total null:lla sen sijaan, että 0.0:lla niin aiheutti nullpointerexceptionii
  * yhteenvetonäkymässä pelkät hinnat eikä koko tekstejä eli ''$6.50'' kun piti olla ''Subtotal: $6.50''

Unit-4 https://developer.android.com/courses/android-basics-kotlin/unit-4

\subsubsection{14.05.2023}

Unit-5 https://developer.android.com/courses/android-basics-kotlin/unit-5

- perus SQL -> menee lähimmä läpi kahlaamiseksi ettei ole mitään android specifiä tms.
- https://github.com/google-developer-training/android-basics-kotlin-bus-schedule-app/issues/76, tosin ohjeessa olikin maininta:
// submitList() is a call that accesses the database. To prevent the
// call from potentially locking the UI, you should use a
// coroutine scope to launch the function. Using GlobalScope is not
// best practice, and in the next step we'll see how to improve this.

\subsubsection{20.05.2023}

- Kotlin extension functionit hämmentää, mutta nyt tekee järkeä aiemmin ihmetelty asia; piti importtaa, jotta tulee funktio käyttöön
  -> Selittää myös miten .isNullOrEmpty() toimii
- Kotlin lambdoissa ei tartte sulkeita niin useampi rivinen näyttää ennemmin siltä, että toinen rivi olisi parametri tjs. Esim. js: test((a) -> teeJotain(a), 1000)

* Project: Forage app:
  * https://stackoverflow.com/questions/55974539/viewmodelscope-launchdispatchers-io-purpose
  * https://github.com/google-developer-training/android-basics-kotlin-forage-app/issues/11

Unit-6 https://developer.android.com/courses/android-basics-kotlin/unit-6

* Android Studiosta löytyy ''Device File Explorer''
* Notifikaatioiden näyttäminen vaatii luvan kysymisen uusimmilla..

KOKONAAN VALMIS!

% TODO END

\subsubsection{Tuntikirjanpito}

Ohjeisessa taulukossa on kirjattu ''Android Bascis in Kotlin''-kurssiin käytetyt tunnit.

\begin{table}[H]
  \centering
  \label{tab:android-basics-in-kotlin-working-hours}
  \begin{tabular*}{\linewidth}{@{\extracolsep{\fill}} l c c c r }
    \textbf{Osio} & \textbf{Päivämäärä} & \textbf{Aloitettu} & \textbf{Lopetettu} & \textbf{Määrä} \\
    \hline
    Unit-1 & 25.01.2023 & 08:10 & 10:40 & 2h 30m \\
    Unit-1 & 25.01.2023 & 11:45 & 14:40 & 2h 55m \\
    Unit-2 & 26.01.2023 & 01:45 & 07:00 & 5h 15m \\
    Unit-2 & 26.01.2023 & 07:50 & 09:50 & 2h 00m \\
    Unit-3 & 26.01.2023 & 09:50 & 11:25 & 1h 35m \\
    Unit-3 & 26.01.2023 & 11:55 & 12:55 & 1h 00m \\
    Unit-3 & 12.05.2023 & 11:40 & 15:30 & 3h 50m \\
    Unit-3 & 12.05.2023 & 16:15 & 17:45 & 1h 30m \\
    Unit-3 & 13.05.2023 & 14:15 & 20:25 & 6h 10m \\
    Unit-4 & 13.05.2023 & 20:25 & 21:05 &    40m \\
    Unit-4 & 14.05.2023 & 15:30 & 16:00 &    30m \\
    Unit-4 & 14.05.2023 & 20:30 & 22:20 & 1h 50m \\
    Unit-5 & 14.05.2023 & 23:10 & 00:50 & 1h 40m \\
    Unit-5 & 20.05.2023 & 06:20 & 08:05 & 1h 45m \\
    Unit-5 & 20.05.2023 & 08:45 & 10:20 & 1h 35m \\
    Unit-6 & 20.05.2023 & 10:20 & 13:00 & 2h 40m \\
    \hline
    \multicolumn{4}{l}{\textbf{Yhteensä}} & \textbf{36h 25m} \\
  \end{tabular*}
\end{table}

\subsection{Android Basics with Compose -kurssi}


\subsubsection{20.05.2023}

% TODO START

Unit-1

* Unit on paluu arvo kun mitään ei palauteta (void)
* Kotlin tukee ''Tekstiä: \$muuttuja''
* Row/Column huomattavasti mukavamman oloisia kun vastaavia kuin flex boxit css:ssä. match\_parentn, startOf.. yms. ollut aina vähän mystisiä tai vähintääkin unohtuu aina miten ne meni

\subsubsection{21.05.2023}

Unit-2

* when tukee myös `in 1..10` ja `is Int`:

\begin{lstlisting}[
  basicstyle=\small,
  label={lst:TODO},
  language=Kotlin,
]
when (x) {
    2, 3, 5, 7 -> println(''x is a prime number between 1 and 10.'')
    in 1..10 -> println(''x is a number between 1 and 10, but not a prime number.'')
    is Int -> println(''x is an integer number, but not between 1 and 10.'')
    else -> println(''x isn't a prime number between 1 and 10.'')
}
\end{lstlisting}

* Elvis operaatio (`?:`) kätevä, pitää vaan muistaa, että se on olemassa
* Kotilinissa ei tarveita `new` kun luodaan instanssi luokasta
* Getterin ja Setterin voi ylikirjoittaa `get()` ja `set()` avulla
  * Mahdollistaa mm. sen että olio.numero++ ei voi kasvaa liian suureksi yms.

\begin{lstlisting}[
  basicstyle=\small,
  label={lst:TODO},
  language=Kotlin,
]
class Test() {
    var i = 2
        set(value) {
            if (value >= 0)
                field = value
        }
}

fun main() {
    val test = Test()
    test.i--
    test.i--
    test.i--
    test.i--
    println(test.i)  // 0
}
\end{lstlisting}

* Setterin sisällä pitää käyttää `field`! Jos yrittää asettaa muuttujan nimen uudelleen (esimerkissä `i`) niin aiheuttaa vaan ikiluupin kun se kutsuu setteria uudelleen
* Setteristä voi tehdä privaten

\begin{lstlisting}[
  basicstyle=\small,
  label={lst:TODO},
  language=Kotlin,
]
var deviceTurnOnCount = 0
    private set
\end{lstlisting}

* funktioon voi viitata `::funktionNimi`
* lamdalla toki ei tartte vitata erikseen
* Lyhyempi keino tehdä luuppi n kertaa: repeat(4) { println(''asdf'') }
* Layout Inspector on juttu

Unit-3

* Singletonin saa tehtyä `object Example { ... }`. Luokan sisälle laittaessa `companion object Example { ... }`, jonka jälkeen object sisällä olevia ominaisuuksia voi kutsua suoraan `Luokka.ominaisuus`.
* Theia?? :D
* println(solarSystem.contains(''Pluto'')), println(''Future Moon'' in solarSystem)
* reduce() aloittaa ekasta kun fold():lle annetaan initial value!
* https://hyperskill.org/tracks
* Foreground + background riittää, että Android Studiolla saa generoitua kaikki tarvittavat laucher imaget

Unit-4

* Resizable emulator alko toimii kun loi uuden??

\subsubsection{22.05.2023}

Unit-5

Unit-6

Unit-7

* Background Task Inspector on juttu

Unit-8

% TODO END

\subsubsection{Tuntikirjanpito}

Ohjeisessa taulukossa on kirjattu ''Android Bascis with Compose''-kurssiin käytetyt tunnit.

\begin{table}[H]
  \centering
  \label{tab:android-basics-with-compose-working-hours}
  \begin{tabular*}{\linewidth}{@{\extracolsep{\fill}} l c c c r }
    \textbf{Osio} & \textbf{Päivämäärä} & \textbf{Aloitettu} & \textbf{Lopetettu} & \textbf{Määrä} \\
    \hline
    Unit-1 & 20.05.2023 & 13:00 & 14:40 & 1h 40m \\
    Unit-2 & 21.05.2023 & 01:55 & 04:40 & 2h 45m \\
    Unit-3 & 21.05.2023 & 04:40 & 08:35 & 3h 55m \\
    Unit-4 & 21.05.2023 & 08:55 & 14:25 & 4h 30m \\
    Unit-5 & 22.05.2023 & 08:40 & 10:20 & 1h 40m \\
    Unit-6 & 22.05.2023 & 10:20 & 11:10 &    50m \\
    Unit-7 & 22.05.2023 & 11:10 & 11:40 &    30m \\
    Unit-8 & 22.05.2023 & 11:50 & 12:45 &    55m \\
    \hline
    \multicolumn{4}{l}{\textbf{Yhteensä}} & \textbf{16h 45m} \\
  \end{tabular*}
\end{table}


\section{Muuta}

Projektin varrella on tullut perehdyttyä yksittäisiin pieniin asioihin, jotka
on esitelty seuraavana.

\subsubsection{Kuvien generointi AI-työkaluilla}

% TODO START

* Tutkittu kuvien generoimista
* Wall-E:ta kokeiltu aiemmin, jotenkin ei tyytyväinen sen tuloksiin
* Midjourney liian suotittu, että voisi ilmais versio edes kokeilla tällä hetkellä
* https://stablediffusionweb.com/ ei oikein antanut mitään, mutta Dreamstudion kautta sai hyviä tuloksia

* Hankala saada uusista kuvista tyyliltään samanlaisia kuin ekasta
* Varsinkin jos haluaa muuttaa muuten kuvaa paljon (poistaa hatun, eriväriset hiukset, vaatteet, vaihtaa sukupuolen piirteitä yms.)
* Tausta lähtee helposti pois: https://removal.ai/

% TODO END

\subsubsection{Tuntikirjanpito}

Ohjeisessa taulukossa on kirjattu muihin opiskelu-luvussa esiteltyihin asioihin käytetyt tunnit.

\begin{table}[H]
  \centering
  \label{tab:other-studing-working-hours}
  \begin{tabular*}{\linewidth}{@{\extracolsep{\fill}} l c c c r }
    \textbf{Tekeminen} & \textbf{Päivämäärä} & \textbf{Aloitettu} & \textbf{Lopetettu} & \textbf{Määrä} \\
    \hline
    Kuvien generointi AI:n avulla & 12.05.2023 & 18:35 & 20:00 & 1h 25m \\
    Kuvien generointi AI:n avulla & 12.05.2023 & 20:25 & 22:45 & 2h 20m \\
    \hline
    \multicolumn{4}{l}{\textbf{Yhteensä}} & \textbf{3h 45m} \\
  \end{tabular*}
\end{table}
