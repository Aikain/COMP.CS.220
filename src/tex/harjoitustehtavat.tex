% TODO START

Tähän tulee tietoa kurssin harjoitustehtävien tekemisestä.

% TODO END

\section{Harjoitus 1}

% TODO START

Tavoitteena selvittää jonkin mobiililaitteen ohjelmointiin vaadittavat asiat.

* Valitse laite joka kiinnostaa( esimerkiksi oma puhelin tai tabletti)

* Merkki, mallii?

* Selvitä laitteen käyttöjärjestelmä versio ja ominaisuudet. (Linkki valmistajan esitteeseen?)

* Selvitä mahdolliset ohjelmointi kielet.

* Selvitä ohjelmointiin tarvittavat työkalut ( + käyttöjärjestelmät?)

* Sisältääkö laite valmistajakohtaisia sovelluksia tai ominaisuuksia? Ja mitä tarkoitetaan (mainos)termillä ”puhdas Android”?

* ”App Store” (laitteen / valmistajan) – onko sinne mahdollista saada itsekoodattu sovellus?

* Selvitä laitteesta löytyvät ominaisuudet (esim GPS) ja pystytäänkö niitä käyttämään alustan tarjoamilla ohjelmointikielillä.

Tavoitteena noin sivun mittainen selvitys, jossa viittauksia lähteisiin. Kaikkia tietoja ei
tarvitse kirjata vaan lähteiden viitteitä voi käyttää apuna(esimerkkinä puhelimen
ominaisuuksia ei kannata kaikkia kirjata)

% TODO END

\section{Harjoitus 2}

Versionhallinnan käyttö on tullut aloitettua lähes 10 vuotta sitten ja on
välttämätön työkalua niin töissä, opiskelussa kuin vapaa-ajan projekteissakin,
joten uskon käytön vähintään peruskäytön hallitsevan. Tämän vuoksi erillistä
harjoitus ohjeen mukaista harjoitus2\_vastaus.txt-tiedostoa ei ole lisätty.

Käytössä on lisäksi ssh-avaimet, kuten harjoituksessa viitattiin. Näiden
lisäksi myös commitit allekirjoitetaan ja käytössä submoduuleja. Näistä oli
tarkemmin selitystä aloitus-osiossa.

Oppimispäiväkirja on myös versionhallinnassa ja PDF-muoto siitä generoidaankin
automaattisesti Github Actioneilla, jotka ovat merkittävä apu asioiden
automatisoinnissa kätevästi versionhallinnan kanssa.

\section{Harjoitus 3}

Android Studion asentamisesa on kerrottu tarkemmin aloitus-osiossa.

Oppimispäiväkirja-repository ja sen myötä myös harjoituksien kansiot
sijaisevat WSL:n tiedostojärjestelmässä. Tähän on keino päästä käsiksi kuten
mm. verkkojakoihin, toisiin kovalevyihin yms. Android Studio ei kuitenkaan
toimi tämän kautta vaan näyttää \ref{fig:android-studio-path-not-writable}
mukaisen virheen, ettei kansio ole kirjoitusoikeuksia.

\begin{figure}[h!]
    \includegraphics[width=\textwidth]{figures/android-studio-path-not-writable.png}
    \caption{Kuvankaappaus Android Studio virheestä kirjoittaa kansioon}
    \label{fig:android-studio-path-not-writable}
\end{figure}

Projektin voi kuitenkin luoda muutoin WSL:n puolelle ja sen jälkeen avata
Android Studiolla. Tästä kuitenkin aiheutuu uusi ongelma; Gradle ei saa
syncattua haluamiaan asioita vaan sanoo ''virhe, yritä uudelleen'' tarkentamatta
mikä virhe. Logeista löytyy hiukan tarkempi virhe:

\begin{displayquote}
GradleConnectionException: Operation result has not been received
\end{displayquote}

Tämän virheen perusteella ei kuitenkaan löydy mitään Android Studioon itseensä
liittyvää vaan kaikki viittaavat kahteen Intellij IDEAn issueeseen (joka toki
tässä tapauksessa on tarpeeksi lähellä), mutta molemmissa puhutaan Android-
lisäosan kytkemisestä pois päältä ja se ei oikein toimi ratkaisuna tähän
tilanteeseen.

On hyvin mahdollista, että virhe johtuu SDK:n sijaitsemisesta Windowsin
puolella kun kaikki muut (projekti, gradle, jdk) sijaisevat WSL:n sisällä.
Kuitenkin kokemuksen pohjalta yleensä fyysisten laitteiden kanssa kommunikointi
WSL:n sisältä käsin on hyvin nihkeää ja jo emulaattorinkin käyttö saattaisi
tuottaa ongelmia, joten ei ole mielekästä lähteä asentamaan Android SDK:ta ja
kaikkea siihen liittyvää WSL:n sisälle.

Pitkälti siis ainut toimiva ja aikaa säästävä ratkaisu on pitää projektit
Windowsin puolella. Kuitenkaan yksittäisten projektin vuoksi en ole siirtämässä
versionhallintaa ja kaikkea muuta configuraatiota toimimaan WSL:n ulkopuolella,
joten tarvitaan keino pitää repo WSL:ssä, mutta harjoitustehtävät Windowsin
puolella. Ensimmäisenä tähän tulee mielenä yksinkertaisesti käyttää symboolista
linkkiä.

\begin{lstlisting}[
    basicstyle=\small,
    label={lst:symbolic-link-harjoitustehtavat},
    language=bash,
]
    # ln -s /mnt/c/Users/Aikain/AndroidStudioProjects/mobiiliohjelmointi-kurssi-harjoitustehtavat/ harjoitustehtavat
\end{lstlisting}

Symboolin linkki näkyy kuitenkin Gitille symboolisena linkkinä eikä normaaleina
tiedostoina, mikä toki on täysin perusteltua mm. tietoturvasyistä. Puolestaan
ns. kovan linkin tekeminen ei ole mahdollista kansioille. Vaihtoehtoisesti
voisi kokeilla linkin tekemistä toisinpäin
\parencite{StackoverflowWSLSymlink}.

\begin{lstlisting}[
    basicstyle=\small,
    label={lst:powershell-link-harjoitustehtavat},
    language=PowerShell,
]
    New-Item -ItemType SymbolicLink -Path ''C:\Aikain\Users\AndroidStudioProjects\mobiiliohjelmointi-harjoitustehtavat'' -Target ''\\wsl$\Debian\home\aikain\projects\COMP.CS.220\harjoitustehtavat''
\end{lstlisting}

Tämä kuitenkin aiheuttaa jälleen samat ongelmat kuin aiemmin Android Studion
kanssa kun yritti käyttää suoraan WSL:n sisällä olevaa sijaintia.

Ei ehkä niin hyvänä ratkaisuna, mutta kuitenkin toistaiseksi toimiva ratkaisu
saadaan käyttämällä bind mountia \parencite{BaeldungBindMounts}.

\begin{lstlisting}[
    basicstyle=\small,
    label={lst:bind-mount-harjoitustehtavat},
    language=bash,
]
    $ mount -o bind /mnt/c/Users/Aikain/AndroidStudioProjects/mobiiliohjelmointi-harjoitustehtavat /home/aikain/projects/koulu/COMP.CS.220/harjoitustehtavat
\end{lstlisting}

Versionhallintaan lisätessä kiinnitin huomiota tarkemmin noihin gradle
tiedostoihin. Jotenkin olen sitä vastaan, että jar-tiedoston laittaisi
versionhallintaan, mutta Gradlen dokumentaatiossa
\parencite{GradleDocsGradleWrapper} sanotaan seuraavasti:

\begin{displayquote}
To make the Wrapper files available to other developers and execution
environments you’ll need to check them into version control. All Wrapper files
including the JAR file are very small in size. Adding the JAR file to version
control is expected. Some organizations do not allow projects to submit binary
files to version control. At the moment there are no alternative options to the
approach.
\end{displayquote}

Toistaiseksi siis laittelen nämä versionhallintaan, mutta olisi hyvä jos tähän
olisi parempi ratkaisu.

Lisäksi .idea-kansion sisältö ei ole oletuksena kokonaan .gitignore:ssa.
Ilmeisesti toki pääosa käyttäjistä käyttää Android Studio:ta, mutta silti IDEn
configuraation laittaminen versionhallintaan ei tunnu oikealta ratkaisulta,
varsinkin kun Android Studio osaa avata projektin ilman niitä ja mitään
projektille merkittäviä asioita ei pitäisi olla IDEn configuraatiossa, jotta
muunmuassa CI/CD toimii ongelmitta kun lähtökohtaisesti ne eivät ole tietoisia
IDEn confeista.

\section{Harjoitus 4}

Tarkoitus painottaa Kotlinilla tekemistä ja tehdä varsinkin projekti
Kotlinilla, joten tässäkin valittu pienempi eli harjoitus 4 javalla tehtäväksi
ja suurempi eli harjoitus 5 Kotlinilla tehtäväksi.

Painotus opiskelussa on ollut Composessa, jota puolestaan ei voi käyttää Javan
kanssa \parencite{StackoverflowComposeInJava}, joten alkuun pääseminen vaatii
muutaman minuutin mietintä tauon, mutta onneksi Android Studio alustaa
valmiiksi uuden projektin luonnin yhteydessä tarvittavat asiat.

\begin{wrapfigure}{r}{0.4\textwidth}
    \includegraphics[width=0.4\textwidth]{figures/exercise-4-attributes.png}
    \caption{Kuvankaappaus Android Studio Design-työkalun attribuuteista}
    \label{fig:exercise-4-attributes}
\end{wrapfigure}

Design-työkalun avulla saa lisättyä helposti tarvittavat osat, mutta jotenkin
graaffisen käyttöliittymän kautta kuvassa \ref{fig:exercise-4-attributes}
näkyvien tietojen täyttäminen ei suju, joten aika vaihtaa Split-tilaan ja
muokata XML:ää suoraan. Tämä sujuukin suoraan helpommin ja jopa yleensä
päänvaivaa aiheuttavat app:layout\_constraintX\_toXOf -asetukset menevät
sujuvasti tällä kertaa.

Aluksi lisään kentät suoraan valmiina olevaan ConstraintLayout:in alle, mutta
tästä aiheutuu se etteivät ne ole nätisti tasattuna yhteen riviin vaan
painikkeen ollessa korkeampi on sen keskikohta selvästi alempana kuin muilla.
Tämän voisi sinääli korjata vain nopeasti marginilla, joka tuntuu huonolta
ratkaisulta. Hetken pohdinnan jälkeen saankin laitettua toisen
ConstraintLayout:in, jonka korkeus on vain sisällön verran ja saan sisällön
keskitettyä pystysuunnassa kyseisen komponentin suhteen. Hiukan paddingia
ConstraintLayout:lle ja numerokentille leveydet niin saadaan siististi yhteen
riviin tasaisin välein kuvan \ref{fig:exercise-4-constraint-layout} mukaisesti.

\begin{figure}[h!]
    \centering
    \includegraphics[width=0.8\textwidth]{figures/exercise-4-constraint-layout.png}
    \caption{Kuvankaappaus komponenttien sijoittelusta}
    \label{fig:exercise-4-constraint-layout}
\end{figure}

UI:n tultua valmiiksi viewBinding-ominaisuuden enablointi ja sen jälkeen
saadaankin helposti lisättyä painikkeelle onClickListener, joka ottaa kentistä
luvut ja laskee ne yhteen ja laittaa tulos-kenttään.

\begin{wrapfigure}{rl}{0.3\textwidth}
    \includegraphics[width=0.3\textwidth]{figures/exercise-4-final.png}
    \caption{Kuvankaappaus lopullisesta sovelluksesta puhelimessa}
    \label{fig:exercise-4-final}
\end{wrapfigure}

Lopulta emulaattorissa toimisen jälkeen vielä hyvä hetki kokeilla, että toimii
sujuvasti puhelimellakin kuten kuvassa \ref{fig:exercise-4-final} näkyy.

Android Studio auttaa vielä nopeasti poistamaan tekstien kovakoodaukset ja
käyttämään strings-resursseja. Parannettavaa näinkin piennessä sovelluksessa
olisi vielä mm. saavutettavuuden suhteen sekä automaattisten testien
tekemisessä. Toki parannettavaa löytyy aina, joten jätetään sovellus tämän
harjoituksen osalta tähän.

\tipbox{
\textbf{Miksi packagen nimen alkuun in.aika?}

Harjoituksissa ja projekteissa esiintyy in.aika-verkkotunnusta. Tämä voi
vaikuttaa hiukan oudolta aluksi, että käytössä Intialainen verkkotunnus. Tähän
on kuitenkin yksinkertainen selitys. Olen pitkään käyttänyt nimimerkkiä
"Aikain" ja kuten jokaisella ''IT-nörtillä'' niin pitäähän minullakin olla
omaverkkotunnus (tai tänä päivänä niitä on jo useita..). Harmikseni aikain.fi
oli varattu, mutta aika.in-verkkotunnus puolestaan oli juuri sopivasti
vapautunut, joten se on ollut nyt 6 vuotta minulla ja näkyy monessa
henkilökohtaisessa projektissani yms.
}

\section{Tuntikirjanpito}

Oheiseen talukkoon on koottu harjoitustehtäviin käytetyt tunnit.

\begin{table}[H]
  \centering
  \label{tab:other-studing-working-hours}
  \begin{tabular*}{\linewidth}{@{\extracolsep{\fill}} l c c c r }
    \textbf{Tekeminen} & \textbf{Päivämäärä} & \textbf{Aloitettu} & \textbf{Lopetettu} & \textbf{Määrä} \\
    \hline
    Harjoitus 2  & 23.05.2023 & 05:15 & 05:25 &    10m \\
    Harjoitus 3  & 23.05.2023 & 05:25 & 08:05 & 2h 40m \\
    Harjoitus 3  & 24.05.2023 & 06:15 & 06:30 &    15m \\
    Harjoitus 4  & 24.05.2023 & 06:30 & 08:40 & 2h 10m \\
    \hline
    \multicolumn{4}{l}{\textbf{Yhteensä}} & \textbf{XXh XXm} \\
  \end{tabular*}
\end{table}
